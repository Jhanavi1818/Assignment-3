%%%%%%%%%%%%%%%%%%%%%%%%%%%%%%%%%%%%%%%%%%%%%%%%%%%%%%%%%%%%%%%
%
% Welcome to Overleaf --- just edit your LaTeX on the left,
% and we'll compile it for you on the right. If you open the
% 'Share' menu, you can invite other users to edit at the same
% time. See www.overleaf.com/learn for more info. Enjoy!
%
%%%%%%%%%%%%%%%%%%%%%%%%%%%%%%%%%%%%%%%%%%%%%%%%%%%%%%%%%%%%%%%


% Inbuilt themes in beamer
\documentclass{beamer}

% Theme choice:
\usetheme{CambridgeUS}

% Title page details: 
\title{Assignment 3} 
\author{Kummitha Jhanavi CS21BTECH11032}
\date{\today}
\logo{\large \LaTeX{}}


\begin{document}

% Title page frame
\begin{frame}
    \titlepage 
\end{frame}

% Remove logo from the next slides
\logo{}


% Outline frame
\begin{frame}{Outline}
    \tableofcontents
\end{frame}

% Blocks frame
\section{Problem Statement}
\begin{frame}{Problem Statement}
    \begin{block}{Ex 13.4 Question 11}
    Two dice are thrown simultaneously. If X denotes the number of sixes, find the expectation of X.

        
    \end{block}
    \end{frame}
    \section{Solution}
    \begin{frame}{Solution}
    \begin{block}
    Two dice are thrown simultaneously. If X denotes the number of sixes, find the expectation of X.
    \begin{align}
        Let S= {1,2,3,4,5,6},n(S)=6
    \end{align}
    \begin{center}
        Let A denotes the number 6
    \end{center}
    \begin{center}
        A = {6}, n(A) = 1, P(A) = $\frac{n(A)}{n(S)}$ = $\frac{1}{6}$
    \end{center}
    \begin{center}
        P(\overline A) = 1-\frac{1}{6} = \frac{5}{6}
    \end{center}
    \begin{center}
         Now n = 2 , r=0,1,2, P(X=0)
         $\implies$ P(\overline A)P(\overline A)= \frac{25}{36}

 P(X = 1) = 2P(A)P(\overline A)=2*\frac{1}{6}*\frac{5}{6} = \frac{10}{36}
    
P(X = 2) = P(A)P(\overline A)=\frac{1}{6}*\frac{1}{6}=\frac{1}{36}

E(X) = 0*P(\overline A)P(\overline A) + 1*P(X =1) + 2*P(X=2) = \frac{12}{36} = \frac{1}{3}
 \end{align}
    \end{block}
        
    \end{frame}
\end{document}
